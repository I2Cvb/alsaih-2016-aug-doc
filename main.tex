% Template for IEEEtransactions
% Adapted by Sik in March 2016 to meet his requirements
%
\documentclass[10pt,conference]{latex/ieeeconf}
%% Latex documents that need direct input
%
% In order to include files without having a clear page using \include*,
% the newclude package is required
\usepackage{newclude}

% Include acroyms
\usepackage[acronym]{glossaries}

% Use biblatex to manage the referencing
%
% \usepackage[style=reading,backend=biber]{biblatex}
% \usepackage[authoryear, round, sort]{natbib}

%% Creating subfigures
%
%  The subcaption package allows for subfloat figure within a single float.
%  This package substitutes the depregated subfigure and subfig packages 
%  allowing to have subfigures within figures, or subtables within table 
%  floats. Subfloats have their own caption, and an optional global 
%  caption. 
%  >> WARNING: some journal templates from Springer and IEETrans might not
%              be compatible with this package forcing to use the 
%              deprecated packages instead.
% \usepackage{subcaption}
%
% IEEETrans is compatible with both subfigure and subfig.
% Subfig is recomended by IEEETrans help documentation
\usepackage{subfigure}

  % contains the latex packages for IEEEtrans
%  The following command loads a graphics package to include images
%  in the document. It may be necessary to specify a DVI driver option,
%  e.g., [dvips], but that may be inappropriate for some LaTeX
%  installations.
\usepackage[]{graphicx}

% Use nameref to cite supporting information files (see Supporting Information section for more info)
\usepackage{nameref,hyperref}


% Clever cross referencing. Using cleverref, instead of writting
% figure~\ref{...} or equation~\ref{...}, only \cref{...} is required.
% The package interprates the references and introduces the figure, fig.,
% equation, eq., etc keywords. \Cref forces first letter capital.
% >> WARNING: This package needs to be loaded after hyperref, math packages,
%             etc. if used.
%             Cleveref is recomended to load late
\usepackage{cleveref}


%% color package
\usepackage[table]{xcolor}
\usepackage{color}
\usepackage{booktabs}
\usepackage{datatool}
\usepackage{multirow}
%\usepackage{arydshln}
%\usepackage{tabularx}
\usepackage{courier}
\usepackage{lscape}
\usepackage{pdflscape}

% packages for checkmark 
\usepackage{bbding}
\usepackage{pifont}
\usepackage{wasysym}
\usepackage{amssymb}

\newcommand{\cmark}{\large \color{green!60!black!80}\ding{51}}
\newcommand{\mmark}{\large {\color{green!60!black!80}\ding{51}}$^{!}$}
\newcommand{\xmark}{\large \color{red!60!black!80}\ding{55}}
\newcommand{\cmarksmall}{\color{green!60!black!80}\ding{51}}
\newcommand{\mmarksmall}{{\color{green!60!black!80}\ding{51}}$^{!}$}
\newcommand{\xmarksmall}{\color{red!60!black!80}\ding{55}}

%% Colorpackage for table
\usepackage{colortbl}
\usepackage{tabularx}
\usepackage{arydshln}

% SI units
\usepackage{siunitx}
% Define the money way to write
\sisetup{
  group-four-digits = true,
  group-separator = {,}
}
\DeclareSIUnit\px{px}

\usepackage{scalefnt,tikz,xifthen}
\usetikzlibrary{fit, patterns, shapes, backgrounds, positioning}
\usetikzlibrary{shadows,arrows,calc,positioning,backgrounds,snakes,decorations.text,decorations.markings,shapes,patterns,fit,chains,decorations.pathreplacing}
        % contains the latex packages
% % To create random text use lipsum
\usepackage{lipsum}

% See the number of line
\usepackage[switch,columnwise]{lineno}
\modulolinenumbers[5]

% Managing TODOES and unfinished figures
\usepackage{todonotes}

% Some packages useful for edition
\usepackage{changebar}
\usepackage{changes}

% Change the top rule
\newcommand*\ctoprule[1]{\addlinespace\cmidrule[\heavyrulewidth]{#1}}
% contains the latex packages
% \usepackage[numbers]{natbib}
%%%%%%%%%%%%%%%%%%%%%%%%%%%%%%%%%%%%%%%%%%%%%%%%%%%%%%%%%%%%%
% Custom setup
%%%%%%%%%%%%%%%%%%%%%%%%%%%%%%%%%%%%%%%%%%%%%%%%%%%%%%%%%%%%%
%>>>> uncomment following for page numbers
% \pagestyle{plain}
%>>>> uncomment following to start page numbering at 301
%\setcounter{page}{301}

% Generates the acronyms list
\makeglossaries

% setup changes to keep track of stuff if needed
% \definechangesauthor[name={our previous work}, color=orange]{old}
% \definechangesauthor[name={fab}, color=red]{fab}
% \definechangesauthor[name={guillaume}, color=blue]{guillaume}
% \definechangesauthor[name={mojh}, color=blue]{mojh}
% \definechangesauthor[name={sik}, color=blue]{sik}
% \setremarkmarkup{(#2)}


        % contains package and variables init.
%!TEX root = ../main.tex

%% Acronym definition example using glossaries package
%% \usepackage{acro} cannot be used with IEEEtrans use Tobias Oetiker’s {acronym}
%% The acronym environment will have a problem with IEEEtran because of the
%% modified IEEE style description list environment. The optional argument of
%% the acronym environment cannot be used to set the width of the longest label.
%% A workaround is to use \IEEEiedlistdecl to accomplish the same thing:
%%
%% \renewcommand{\IEEEiedlistdecl}{\IEEEsetlabelwidth{S
%% ONET}}
%% \begin{acronym}
%% .
%% .
%% \end{acronym}
%% \renewcommand{\IEEEiedlistdecl}{\relax}% reset back

\newacronym{us}{US}{Ultra-Sound}
\newacronym{cad}{CAD}{Computer Aided Diagnosis}
\newacronym{dm}{DM}{Digital Mammography}
\newacronym{gt}{GT}{Ground Truth}
\newacronym{ml}{ML}{Machine Learning}
\newacronym{acm}{ACM}{Active Contour Model}
\newacronym{crf}{CRFs}{Conditional Random Fields}
\newacronym{mrf}{MRFs}{Markov Random Fields}
\newacronym{cv}{CV}{Computer Vision}
\newacronym{icm}{ICM}{Iterated Conditional Modes}
\newacronym{sa}{SA}{Simulate Anealing}
\newacronym{gc}{GC}{Graph-Cuts}
\newacronym{birads}{BI-RADS}{Breast Imaging-Reporting and Data System}
\newacronym{mad}{MAD}{Median Absolute Deviation}
\newacronym{qc}{QC}{Quadratic-Chi}
\newacronym{sift}{SIFT}{Self-Invariant Feature Transform}
\newacronym{bof}{BoF}{Back-of-Features}
\newacronym{acr}{ACR}{American College of Radiology}
\newacronym{fa}{FA}{Fibro-Adenoma}
\newacronym{dic}{DIC}{Ductal Inflating Carcinoma}
\newacronym{ilc}{ILC}{Inflating Lobular Carcinoma}
\newacronym{fpr}{FPR}{False Positive Ratio}
\newacronym{fnr}{FNR}{False Negative Ratio}
\newacronym{fn}{FN}{False Negative}
\newacronym{fp}{FP}{False Positive}
\newacronym{rbf}{RBF}{Radial Basis Function}
\newacronym{dr}{DR}{Diabetic Retinopathy}
\newacronym{dme}{DME}{Diabetic Macular Edema}
\newacronym{oct}{OCT}{Optical Coherence Tomography}
\newacronym{sdoct}{SD-OCT}{Spectral Domain OCT}
\newacronym{amd}{AMD}{Age-related Macular Degeneration}
\newacronym{hog}{HOG}{Histogram of Oriented Gradients}
\newacronym{svm}{SVM}{Support Vector Machines}
\newacronym{bow}{BoW}{Bag-of-Words}
\newacronym{rf}{RF}{Random Forest}
\newacronym{tp}{TP}{True Positive}
\newacronym{tn}{TN}{True Negative}
\newacronym{roc}{ROC}{Receiver Operating Characteristic}
\newacronym{auc}{AUC}{Area Under the Curve}
\newacronym{lbp}{LBP}{Local Binary Patterns}
\newacronym{pca}{PCA}{Principal Component Analysis}
\newacronym{nlm}{NLM}{Non-Local Means}
\newacronym{lopocv}{LOPO-CV}{Leave-One-Patient Out Cross-Validation}
\newacronym{lbptop}{LBP-TOP}{LBP from Three Orthogonal Planes}
\newacronym{se}{SE}{Sensitivity}
\newacronym{sp}{SP}{Specificity}
\newacronym{sw}{P}{patch}
\newacronym{nn}{NN}{Nearest Neighbor}
\newacronym{gb}{GB}{Gradient Boosting}
\newacronym{lr}{LR}{Logistic Regression}
\newacronym{adb}{AdB}{AdaBoost}
\newacronym{acc}{ACC}{Accuracy}
\newacronym{f1}{F1}{F1-score}
\newacronym{nf}{NF}{non-flatten}
\newacronym{f}{F}{flatten}
\newacronym{fal}{F+A}{flatten-aligned}
\newacronym{fac}{F+A+C}{flatten-aligned-cropped}
\newacronym{rpe}{RPE}{Retinal Pigment Epithelium}
\newacronym{gmm}{GMM}{Gaussian Mixture Model}
\newacronym{voi}{VOI}{volume of interest}
\newacronym{glcm}{GLCM}{Gray-level co-occurrence matrix}
\newacronym{seri}{SERI}{Singapore Eye Research Institute}
\newacronym{bm3d}{BM3D}{Block Matching 3D filtering}
\newacronym{ltpocv}{LTPO-CV}{Leave-Two-Patient Out Cross-Validation}      % contains the acronims

%% Select inputing only one part of the document
%\includeonly{content/intro/intro}   % the file wihtout .tex
%\includeonly{content/other/other_content}

% \addbibresource{./content/lit_review.bib}
% \addbibresource{./content/biblatex-examples.bib}

%% Include all macros below

\newcommand{\lorem}{{\bf LOREM}}
\newcommand{\ipsum}{{\bf IPSUM}}

%% END MACROS SECTION

\begin{document}
%%Command to thanks
% \IEEEoverridecommandlockouts  

%% Command to meet the printer requirements
\overrideIEEEmargins



% Article Title
%\title{\LARGE \bf 
% Classification of SD-OCT volumes with multi pyramids, LBP and HOG descriptors: application to DME detections}

% Authors 
% Author(s) Name(s)
\def \AuthorA{C.K.~Alsaih}
\def \AuthorB{G.~Lema\^itre}
\def \AuthorC{J.~Massich Vall}
\def \AuthorD{M.~Rastgoo}
\def \AuthorE{D.~Sidib\'e}
\def \AuthorF{T.Y.~Wong}
\def \AuthorG{E.~Lamoureux}
\def \AuthorH{D.~Milea}
\def \AuthorI{C.~Leung}
\def \AuthorJ{F.~M\'eriaudeau}

% Author(s) Email(s)
\def \AuthorAemail{g.lemaitre58@gmail.com}

% Institution(s) Name(s)
\def \InstitutionA{LE2I UMR6306, CNRS, Arts et M\'etiers, Univ. Bourgogne Franche-Comt\'e,\\ 12 rue de la Fonderie, 71200 Le Creusot, France}
\def \InstitutionB{Singapore Eye Research Institute, Singapore National Eye Center, Singapore}
\def \InstitutionC{Centre for Intelligent Signal and Imaging Research (CISIR), Electrical \& Electronic Engineering Department,\\ Universiti Teknologi Petronas, 32610 Seri Iskandar, Perak, Malaysia}

% Article title
\title{\LARGE \bf 
 Classification of SD-OCT volumes with multi pyramids, LBP and HOG descriptors: application to DME detections}

\author{\AuthorA\authorrefmark{1}, \AuthorB\authorrefmark{1}, \AuthorC\authorrefmark{1}, \AuthorD\authorrefmark{1}, \AuthorE\authorrefmark{1},\\
 \AuthorF\authorrefmark{2}, \AuthorG\authorrefmark{2}, \AuthorH\authorrefmark{2}, \AuthorI\authorrefmark{2},\\
 \AuthorJ\authorrefmark{3}\\
\authorblockA{\authorrefmark{1}\InstitutionA}
\authorblockA{\authorrefmark{2}\InstitutionB}
\authorblockA{\authorrefmark{3}\InstitutionC}
\authorblockA{\authorrefmark{5}Corresponding author: \AuthorAemail}
}
             % contains the Title and Autor info

\maketitle

% Please keep the abstract below 300 words
\begin{abstract}
This paper deals with the automated detection of \gls{dme} on \gls{oct} volumes.
Our method considers a generic classification pipeline with preprocessing for noise removal and flattening of each B-Scan.
Features such as \gls{hog} and \gls{lbp} are extracted and combined to create a set of different feature vectors which are fed to a linear SVM Classifier.
Experimental Results show a promising sensitivity/specificity of 0.75/0.875 on a challenging dataset.
\end{abstract}

\begin{keywords}
  \glsresetall % reset the acronyms from the abstract
  \gls{dme},
  \gls{sdoct},
  \gls{ml},
  benchmark,
\end{keywords}

% \linenumbers

%% Incldue the content without .tex extension
\glsresetall % reset the acronyms from the abstract
\include*{content/intro/intro}          % the file wihtout .tex
\include*{content/survey/background}
\include*{content/method/dataset}
\begin{table*}
  \centering
  \caption{Summary of the classification performance in terms of \gls{se} and \gls{sp} in (\%).}
  \begin{tabular}{l c c c c c c}
    \toprule
    & Lemaitre~\emph{et~al.}~\cite{Lemaintre2015miccaiOCT}
    & Sankar~\emph{et~al.}~\cite{sankar2016classification}
    & Alsaih~\emph{et~al.}~\cite{Alsaih2016apr-repoICPR}
    & Srinivasan~\emph{et~al.}~\cite{Srinivasan2014}
    & Liu~\emph{et~al.}~\cite{Liu2011}
    & Venhuizen~\emph{et~al.}~\cite{Venhuizen2015}
    \\ \midrule

    %         & Lemai.& Sankar  & Alsaih & Srin. & Liu   & Venhu.
    \gls{se}  & 87.5  & 81.3    & 75.0   & 68.8  & 68.8  & 61.5  \\
    \gls{sp}  & 75.0  & 62.5    & 87.5   & 93.8  & 93.8  & 58.8  \\
    \bottomrule
    \end{tabular}
    \label{tab:summary_results}
\end{table*}

%%% Local Variables:
%%% mode: latex
%%% TeX-master: "../../main"
%%% End:

\include*{content/method/method}
\include*{content/results/results}
% \section{Discussion}\label{sec:discussion}
\section{Conclusion and Further work}\label{sec:conclusion}
The work here presented states the relevance of developing methodologies to automatically differentiate \gls{dme} \emph{vs.} normal \gls{sdoct} scans.
This article offers an overview of the state-of-the-art of \gls{dme} detection and provides a public benchmarking to facilitate further studies.
In this regard, there are two crucial aspects to improve the work here presented:
(i) enlarge the dataset.
%(ii) reach out for the other authors of those methods in Sect.\,\ref{sec:review} that could not be included in Sect.\,\ref{sec:exp} because our implementation could not be tested against the original data or we did not achieve the original results.
(ii) reach out to other authors in order to enlarge this benchmark with additional methods and improve the existing approaches.

% \nolinenumbers

%\section*{References}
% Either type in your references using
% \begin{thebibliography}{}
% \bibitem{}
% Text
% \end{thebibliography}
%
% OR
%
% Compile your BiBTeX database using our plos2015.bst
% style file and paste the contents of your .bbl file
% here.

% % Imports the bibliography file "sample.bib"
% \bibliography{sample}
\bibliographystyle{IEEEtran}
\IEEEtriggeratref{17}
\bibliography{content/bib/literature_review,content/bib/retinopathy-bibtex/retinopathy-bibtex,content/bib/i2cvb-retinopathy-bibtex/i2cvb-retinopathy-bibtex}

\end{document}
