% Template for IEEEtransactions
% Adapted by Sik in March 2016 to meet his requirements
%
\documentclass[10pt,conference]{latex/ieeeconf}
%% Latex documents that need direct input
%
\input{./latex/filesystem/ieee_packages.tex}  % contains the latex packages for IEEEtrans
\input{./latex/filesystem/package.tex}        % contains the latex packages
% \input{./latex/filesystem/package_edition.tex}% contains the latex packages
% \usepackage[numbers]{natbib}
\input{latex/filesystem/fileSetup.tex}        % contains package and variables init.
\input{content/acronym_definition.tex}      % contains the acronims

%% Select inputing only one part of the document
%\includeonly{content/intro/intro}   % the file wihtout .tex
%\includeonly{content/other/other_content}

% \addbibresource{./content/lit_review.bib}
% \addbibresource{./content/biblatex-examples.bib}

%% Include all macros below

\newcommand{\lorem}{{\bf LOREM}}
\newcommand{\ipsum}{{\bf IPSUM}}

%% END MACROS SECTION

\begin{document}
%%Command to thanks
% \IEEEoverridecommandlockouts  

%% Command to meet the printer requirements
\overrideIEEEmargins



% Article Title
%\title{\LARGE \bf 
% Classification of SD-OCT volumes with multi pyramids, LBP and HOG descriptors: application to DME detections}

% Authors 
% Author(s) Name(s)
\def \AuthorA{Khaled Alsaih}
\def \AuthorB{Guillaume Lema\^{i}tre}
\def \AuthorC{Join Massich Vall}
\def \AuthorD{Mojdeh Rastgoo}
\def \AuthorE{D\'esir\'e Sidib\'e}
\def \AuthorF{Tien Y Wong}
\def \AuthorG{Ecosse Lamoureux}
\def \AuthorH{Dan Milea}
\def \AuthorI{Carol Y Cheung}
\def \AuthorJ{Fabrice M\'eriaudeau}

% Author(s) Email(s)
\def \AuthorAemail{g.lemaitre58@gmail.com}

% Institution(s) Name(s)
\def \InstitutionA{LE2I UMR6306, CNRS, Arts et M\'etiers, Univ. Bourgogne Franche-Comt\'e,\\ 12 rue de la Fonderie, 71200 Le Creusot, France}
\def \InstitutionB{Singapore Eye Research Institute, Singapore National Eye Center, Singapore}
\def \InstitutionC{Centre for Intelligent Signal and Imaging Research (CISIR), Electrical \& Electronic Engineering Department,\\ Universiti Teknologi Petronas, 32610 Seri Iskandar, Perak, Malaysia}

% Article title
\title{\LARGE \bf 
 Classification of SD-OCT volumes with multi pyramids, LBP and HOG descriptors: application to DME detections}

\author{\AuthorA\authorrefmark{1}, \AuthorB\authorrefmark{1}, \AuthorC\authorrefmark{1}, \AuthorD\authorrefmark{1}, \AuthorE\authorrefmark{1},\\
 \AuthorF\authorrefmark{2}, \AuthorG\authorrefmark{2}, \AuthorH\authorrefmark{2}, \AuthorI\authorrefmark{2},\\
 \AuthorJ\authorrefmark{3}\\
\authorblockA{\authorrefmark{1}\InstitutionA}
\authorblockA{\authorrefmark{2}\InstitutionB}
\authorblockA{\authorrefmark{3}\InstitutionC}
\authorblockA{\authorrefmark{5}Corresponding author: \AuthorAemail}
}
             % contains the Title and Autor info

\maketitle

% Please keep the abstract below 300 words
\begin{abstract}
This paper deals with the automated detection of \gls{dme} on \gls{oct} volumes.
Our method considers a generic classification pipeline with preprocessing for noise removal and flattening of each B-Scan.
Features such as \gls{hog} and \gls{lbp} are extracted and combined to create a set of different feature vectors which are fed to a linear SVM Classifier.
Experimental Results show a promising sensitivity/specificity of 0.75/0.875 on a challenging dataset.
\end{abstract}

\begin{keywords}
  \glsresetall % reset the acronyms from the abstract
  \gls{dme},
  \gls{sdoct},
  \gls{ml},
  benchmark,
\end{keywords}

% \linenumbers

%% Incldue the content without .tex extension
\glsresetall % reset the acronyms from the abstract
\include*{content/intro/intro}          % the file wihtout .tex
\include*{content/survey/background}
\include*{content/method/dataset}
\input{./content/results/table_summary}
\include*{content/method/method}
\include*{content/results/results}
% \section{Discussion}\label{sec:discussion}
\section{Conclusion and Further work}\label{sec:conclusion}
The work here presented states the relevance of developing methodologies to automatically differentiate \gls{dme} \emph{vs.} normal \gls{sdoct} scans.
This article offers an overview of the state-of-the-art of \gls{dme} detection and provides a public benchmarking to facilitate further studies.
In this regard, there are two crucial aspects to improve the work here presented:
(i) enlarge the dataset.
%(ii) reach out for the other authors of those methods in Sect.\,\ref{sec:review} that could not be included in Sect.\,\ref{sec:exp} because our implementation could not be tested against the original data or we did not achieve the original results.
(ii) reach out to other authors in order to enlarge this benchmark with additional methods and improve the existing approaches.

% \nolinenumbers

%\section*{References}
% Either type in your references using
% \begin{thebibliography}{}
% \bibitem{}
% Text
% \end{thebibliography}
%
% OR
%
% Compile your BiBTeX database using our plos2015.bst
% style file and paste the contents of your .bbl file
% here.

% % Imports the bibliography file "sample.bib"
% \bibliography{sample}
\bibliographystyle{IEEEtran}
\IEEEtriggeratref{17}
\bibliography{content/bib/literature_review,content/bib/retinopathy-bibtex/retinopathy-bibtex,content/bib/i2cvb-retinopathy-bibtex/i2cvb-retinopathy-bibtex}

\end{document}
